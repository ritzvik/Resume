%%%%%%%%%%%%%%%%%%%%%%%%%%%%%%%%%%%%%%%%%
% Medium Length Professional CV
% LaTeX Template
% Version 2.0 (8/5/13)
%
% This template has been downloaded from:
% http://www.LaTeXTemplates.com
%
% Original author:
% Trey Hunner (http://www.treyhunner.com/)
%
% Important note:
% This template requires the resume.cls file to be in the same directory as the
% .tex file. The resume.cls file provides the resume style used for structuring the
% document.
%
%%%%%%%%%%%%%%%%%%%%%%%%%%%%%%%%%%%%%%%%%

%----------------------------------------------------------------------------------------
%	PACKAGES AND OTHER DOCUMENT CONFIGURATIONS
%----------------------------------------------------------------------------------------

\documentclass{resume} % Use the custom resume.cls style

\usepackage{hyperref}
\usepackage[left=0.75in,top=0.6in,right=0.75in,bottom=0.6in]{geometry} % Document margins

  % If using pdflatex:
\usepackage[utf8]{inputenc}
\usepackage[T1]{fontenc}
\usepackage[default]{lato}
\usepackage{fontawesome}
\usepackage{setspace}

\usepackage{hyperref}
\hypersetup{
    colorlinks=true,
    linkcolor=blue,
    filecolor=magenta,      
    urlcolor=cyan,
}



\newcommand{\tab}[1]{\hspace{.2667\textwidth}\rlap{#1}}
\newcommand{\itab}[1]{\hspace{0em}\rlap{#1}}
\name{Ritwik Saha} % Your name
\address{\faMapMarker ~ Mandi, Himachal Pradesh, India} % Your address
%\address{123 Pleasant Lane \\ City, State 12345} % Your secondary addess (optional)
\address{\faPhone ~ (+91)~7838958076 ~ \faEnvelope ~ b16110@students.iitmandi.ac.in ~ \faLinkedin ~ www.linkedin.com/in/ritwik-saha ~ \faGithub ~ github.com/ritzvik} % Your phone number and email

\begin{document}

%----------------------------------------------------------------------------------------
%	EDUCATION SECTION
%----------------------------------------------------------------------------------------

\begin{rSection}{Education}

{\bf Bachelor of Technology(Electrical Engineering)} \hfill {\em 2016 - 2020} 
\\ Indian Institute of Technology, Mandi \hfill { Overall GPA: 7.6/10}
\\ School of Computing and Electrical Engineering

\smallskip

{\bf CBSE(Higer Secondary)} \hfill {\em 2016} 
\\ D.A.V. Public School, Shreshtha Vihar, Delhi \hfill { Percentage: 94.4\%}

\smallskip

{\bf CBSE(Matriculation)} \hfill {\em 2014} 
\\ D.A.V. Public School, Shreshtha Vihar, Delhi \hfill { CGPA: 10.0}

\end{rSection}
%----------------------------------------------------------------------------------------
%	TECHNICAL STRENGTHS SECTION
%----------------------------------------------------------------------------------------

\begin{rSection}{Technical Strengths}

\begin{tabular}{ @{} >{\bfseries}l @{\hspace{6ex}} l }
Computer Languages &  C/C++, Python, MATLAB \\
Deep Learning Frameworks & Keras, TensorFlow \\
Containerization \& Orchestration & Docker, Kubernetes, KNative \\
IoT & Arduino, RaspberryPi \\
Graph Databases & Neo4J, ArangoDB \\
Blockchain & Ethereum, Solidity, Web3 \\
Version Control & Git, GitHub

\end{tabular}

\end{rSection}

%----------------------------------------------------------------------------------------
%	WORK EXPERIENCE SECTION
%----------------------------------------------------------------------------------------

\begin{rSection}{Experience}

\begin{rSubsection}{Siemens Technology \& Services Pvt. Ltd.}{June 2019 - August 2019}{Research Intern}{}
\item Set up infrastructure for running serverless applications.
\item Leveraged Docker, Kubernetes and KNative for setting up along with various network management framework like Istio.
\item Benchmarked the performance of the serverless setup under various load conditions and network topology.
\item Documented the relevant code and procedures. 
\end{rSubsection}

%------------------------------------------------

\begin{rSubsection}{Siemens Technology \& Services Pvt. Ltd.}{December 2018 - Febraury 2019}{Research Intern}{}
\item Set up blockchain infrastructure with help of Private Ethereum.
\item Benchmarked the transaction performance, system resource usage and network performance under a variety of conditions like variable block times, transaction loads, block size etc.
\item Designed and partly implemented a supply chain solution on blockchain infrastructure.
\item Tried out and partially benchmarked various competing solutions for blockchain like Tendermint
\item Documented the relevant codebase and procedures.
\end{rSubsection}

%------------------------------------------------

\begin{rSubsection}{Indian Institute of Technology, Mandi}{June 2017 - January 2018}{Research Intern}{}
\item Organized and created dataset for IMD rainfall data and soil data from ISRO.
\item Ran Machine Learning algorithms on the dataset to predict landslide predictions.
\item Published the results in a paper presented at International Conference for Machine Learning and Data Science, 2018 and published on IEEE Xplore.
\end{rSubsection}

\end{rSection}


%	EXAMPLE SECTION
%----------------------------------------------------------------------------------------

%----------------------------------------------------------------------------------------
\begin{rSection}{Relevant Courses}
\itab{\textbf{Core Courses}} \tab{}  \tab{\textbf{Computer Science Courses}}
\\ \itab{Signals \& Systems } \tab{}  \tab{Data Structures and Algorithms}
\\ \itab{Control Theory} \tab{}  \tab{Communicating and Distributed Processes} 
\\ \itab{Communication Theory} \tab{}  \tab{Deep Learning} 
\\ \itab{Electromechanics} \tab{} \tab{Artificial Intelligence}
\\ \itab{Network Theory} \tab{} \tab{Computer Organization}
% \\ \itab{Process Control (ongoing)} \tab{} \tab{Electrodynamics}
\newline
\newline
\itab{\textbf{Other Relevant Courses}} 
\\ \itab{Probablity, Statistics \& Random Processes } 
\\ \itab{Linear Algebra} 
\\ \itab{Mathematics for Engineers} 


\end{rSection}




\begin{rSection}{POSITION OF RESPONSIBILITIES}

\begin{rSubsection}{\href{https://socis.esa.int/projects}{Summer of Code in Space }}{June 2019 - September 2019}{Mentor}{European Space Agency}
\item Assigned as project mentor for in SOCIS(Summer of Code in Space) organized by the European Space Agency(ESA).
\item The project is about extending the EinsteinPy library to support symbolic calculations in General Relativity.
\item The project is fiscally sponsored by European Space Research and Technology Centre(ESTEC) wing of ESA.
\end{rSubsection}

%------------------------------------------------

\begin{rSubsection}{Indian Youth Delegation to China}{July 2018}{Delegate}{Ministry of Sports \& Youth Affairs, Govt. of India}
\item Represented Indian contingent as a delegate in Indian Youth Delegation to China - 2018
\item Interacted with top officials within the Chinese Government and the Chinese youth.
\end{rSubsection}

%------------------------------------------------

\begin{rSubsection}{Exodia(Tech-Cult Fest of IIT-Mandi)}{April 2018}{Video Design Coordinator}{IIT Mandi}
\item Was tasked with the job of creating teasers, trailers and promo videos of the fest Exodia.
\item This improved my video editing skills on Adobe Premiere Pro \& After Effects.
\item \href{https://www.youtube.com/playlist?list=PLstTANwGhotvZNHDSR__iPnlhg73xYvq4}{Link} to the videos.
\end{rSubsection}

\end{rSection}



%----------------------------------------------------------------------------------------
\begin{rSection}{PROJECTS} 

\begin{rSubsection}{The EinsteinPy Project}{February 2019 - Present}{}{}
\item Founder of EinsteinPy - An Open-Source Python Library for General Relativity
\item Partly sponsored by European Space Agency(ESA).
\item Soon to be sub-organization under OpenAstronomy
\item \href{https://github.com/einsteinpy/einsteinpy}{GitHub Repository}
\end{rSubsection}

\begin{rSubsection}{\href{https://github.com/ritzvik/Miscellaneous-Projects/blob/master/design_practicum.pdf}{Exoskeleton for Motion Assistance}}{January 2018 - April 2018}{}{}
\item Designed \& Implemented an Exoskeleton intended for military purposes.
\item The Exoskeleton enabled the user to lift weights upto 40 kgs without any locomotive hindrance.
\item The Project won 1\textsuperscript{st} prize in 2018 edition of the Design Practicum Curriculum.
\end{rSubsection}

\begin{rSubsection}{\href{https://github.com/ritzvik/Miscellaneous-Projects/blob/master/techmeet.pdf}{Safety Device for Fishing Vessels}}{November 2017 - January 2018}{}{}
\item Designed \& Implemented a low-cost solution for small vessels to avoid
collision with big ships at night.
\item The project estensilvely used arrays of Arduino and RaspberryPi along with GPS and BlueTooth modules. This was a core IoT project.
\item The Project won 5\textsuperscript{th} prize in 2018 edition of the Inter-IIT Tech Meet held at IIT Madras.
\end{rSubsection}

\begin{rSubsection}{\href{https://github.com/ritzvik/DL-Video-Colorization}{Video Colorization using Deep Learning}}{April 2019}{}{}
\item Used deep learning to colorize black and white videos.
\item Used Autoencoder like networks with various skip connections and CNN blocks.
\item Poposed an time-series aware approach to colorize videos.
\end{rSubsection}

\end{rSection}




%--------------------------------------------------
\begin{rSection}{PUBLICATIONS} 

\item K. Agrawal, Y. Baweja, D. Dwivedi, R. Saha et
al., "A Comparison of Class Imbalance Techniques
for Real-World Landslide Predictions",
in 2017 International Conference on Machine
Learning and Data Science (MLDS), DOI
10.1109/MLDS.2017.21. Published by IEEE.

\end{rSection}



%----------------------------------------------------
\begin{rSection}{INTERESTS}
\item Deep Learning
\item Blockchain
\item Distributed Computing
\item Data Structures \& Algorithms
\item Computer Architecture
\item Differential Geometry \& General Relativity
\end{rSection}
\end{document}
